\documentclass[cyan,pad,cn]{elegantnote}

\usepackage{amsmath}
\usepackage{listings}
\usepackage{xcolor}
\usepackage{amssymb}


\usepackage{tikz}

\usepackage{pgfplots}
\pgfplotsset{compat=1.17}

\definecolor{CPPLight}  {HTML} {686868}
\definecolor{CPPSteel}  {HTML} {888888}
\definecolor{CPPDark}   {HTML} {262626}
\definecolor{CPPBlue}   {HTML} {4172A3}
\definecolor{CPPGreen}  {HTML} {487818}
\definecolor{CPPBrown}  {HTML} {A07040}
\definecolor{CPPRed}    {HTML} {AD4D3A}
\definecolor{CPPViolet} {HTML} {7040A0}
\definecolor{CPPGray}  {HTML} {B8B8B8}
\lstset{
    columns=fixed,       
    numbers=left,                                        % 在左侧显示行号
    frame=none,                                          % 不显示背景边框
    backgroundcolor=\color[RGB]{245,245,244},            % 设定背景颜色
    keywordstyle=\color[RGB]{40,40,255},                 % 设定关键字颜色
    numberstyle=\footnotesize\color{darkgray},           % 设定行号格式
    commentstyle=\it\color[RGB]{0,96,96},                % 设置代码注释的格式
    stringstyle=\rmfamily\slshape\color[RGB]{128,0,0},   % 设置字符串格式
    showstringspaces=false,                              % 不显示字符串中的空格
    language=c++,                                        % 设置语言
    morekeywords={alignas,continute,friend,register,true,alignof,decltype,goto,
    reinterpret_cast,try,asm,defult,if,return,typedef,auto,delete,inline,short,
    typeid,bool,do,int,signed,typename,break,double,long,sizeof,union,case,
    dynamic_cast,mutable,static,unsigned,catch,else,namespace,static_assert,using,
    char,enum,new,static_cast,virtual,char16_t,char32_t,explict,noexcept,struct,
    void,export,nullptr,switch,volatile,class,extern,operator,template,wchar_t,
    const,false,private,this,while,constexpr,float,protected,thread_local,
    const_cast,for,public,throw,std},
    emph={map,set,multimap,multiset,unordered_map,unordered_set,
    unordered_multiset,unordered_multimap,vector,string,list,deque,
    array,stack,forwared_list,iostream,memory,shared_ptr,unique_ptr,
    random,bitset,ostream,istream,cout,cin,endl,move,default_random_engine,
    uniform_int_distribution,iterator,algorithm,functional,bing,numeric,},
    emphstyle=\color{CPPViolet}, 
}

\title{
    补题笔记·CodeForces\#735(Div.2)
}
\author{唐冬   \LaTeX}

\begin{document}

\maketitle

\section{
    A.Cherry
}

\subsection{题意}
\paragraph{
    原题\href{https://codeforces.com/contest/1554/problem/A}{链接}
    \\给出一个正整数数列,找出一个连续子序列,使得序列中最大值与最小值乘积最大。
}

\subsection{思路}
\paragraph{
    这次cf比赛感觉就像是数学专场,看懂题目相对容易,代码也比较短,但想出题解的难度简直没谁了,下次一定好好学数论QAQ。
    \\这个题表面上说是找区间,但通过观察可以发现,只有区间内最大值和最小值影响结果,和其他值无关。
    \\现在,我们想办法找出两个最值。易证:正整数序列相邻两数乘积为任意区间内最大最小值乘积的最大值。
    \\写易证会被扣分(?),那我尝试证明一下。首先在曲线上选两个连续的点,构成一个最小区间,最大值为A点,最小值为B点。
    \\我们在左边取一点C构成新区间[C,A],发现最小值变小了;如果在右边取一点D构成区间[B,D],发现最大值变大,但最小值却不是最佳(向右移后变大)。
    \\所以,遍历整个数组,找出相邻两数乘积的最大值就是题解。\\
}

\begin{center}
    \usetikzlibrary {datavisualization.formats.functions} \begin{tikzpicture}[baseline]
        \datavisualization [ school book axes, visualize as line ] data [format=function] {
            var x : interval [0*pi:6];
            func y = sin(\value x*(1/3) r)*3;
          }
        info {
        \draw [red] (visualization cs: x={(.945*pi)}, y=2.5) circle [radius=1pt]
        node [above,font=\footnotesize] {A}; 
        \draw [red] (visualization cs: x={(.69*pi)}, y=2) circle [radius=1pt]
        node [above,font=\footnotesize] {B}; 
        \draw [red] (visualization cs: x={(1.5*pi)}, y=3) circle [radius=1pt]
        node [above,font=\footnotesize] {D}; 
        \draw [red](visualization cs: x={(0.495*pi)}, y=1.5) circle [radius=1pt]
        node [above,font=\footnotesize] {C};
        };
    \end{tikzpicture}
    \\图1
\end{center}



\subsection{代码}

\begin{lstlisting}
    #include"bits/stdc++.h"

    using namespace std;
    
    const int nmax=1e5+10;
    
    long long val[nmax],l_val[nmax],r_val[nmax];
    
    int main(){
        freopen("input.txt","r",stdin);
        int T;
        cin>>T;
        while(T--){
            int n;
            cin>>n;
            for(int i=0;i<n;i++){
                cin>>val[i];
            }
            long long ans=-1,max_l=val[0];
            for(int i=1;i<=n;i++){
                max_l=max(max_l,val[i-1]);
                memcpy(r_val,val,sizeof(val));
                sort(l_val,l_val+i);
                sort(r_val+i,r_val+n);
                ans=max(l_val[i-1]*r_val[i],ans);
            }
            cout<<ans<<endl;
        }
    }
\end{lstlisting}

\section{
    B. Cobb
}

\subsection{题意}

\paragraph{原题\href{https://codeforces.com/contest/1554/problem/B}{链接}
    \\给出数列\(a_{1},a_{2},a_{3},...,a_{n}\),求全部数对\((a_{i},a_{j})\)中\(i\cdot j-k\cdot (a_{i}|a_{j})\)的最大值(|为\href{https://baike.baidu.com/item/位运算/6888804?fr=aladdin}{or运算})。
    \\数据范围:\(2 \leqslant  n \leqslant  10^5\),\(1 \leqslant  k \leqslant min(n,100)\)。
}

\subsection{思路}

\begin{paragraph}
    全遍历时间复杂度\(\mathcal{O}(n^2)\)会超时,我们考虑优化。
    设
    \begin{equation}
        f(i,j)=i\cdot j-k\cdot (a_{i}|a_{j}),i<j
    \end{equation}
    由于\(i,j\)的范围,\((i|j)\)的最大值是小于等于\(2n\)的,只考虑\(i\cdot j\)部分,它的最大值等于\(n\cdot (n-1)\),所以,最后两个数的最小值可能是:
    \begin{equation}
        \begin{split}
        f_{min}(n-1,n)&=n(n-1)-2kn\\
        &=n^2-2kn-n
        \end{split}
    \end{equation}
    任意数对的最大值为:
    \begin{equation}
        f_{max}(i,j)=i\cdot j
    \end{equation}
    由公式(3)得:
    \begin{equation}
            f_{max}(i,n) = i\cdot n 
    \end{equation}
    令\(f(i,n)>f(n-1,n)\),则:
    \begin{equation}
        \begin{split}
        f_{max}(i,n) &>f_{min}(n-1,n)\\
        i\cdot n &> n^2-2kn-n \\
        i &> n-2k-1
        \end{split}
    \end{equation}
    从上面的推导可以得出,要想得出最大值,\(i,j\)必须大于\(n-2k-1\),也就是说,只需分别遍历2k次就能求出答案,时间复杂度变为\( \mathcal{O}(k^2)\),而题目中k的数据范围远小于n,所以能在规定时间内完成遍历。
\end{paragraph}

\subsection{代码}
\begin{lstlisting}
    #include"bits/stdc++.h"

using namespace std;

const int nmax=1e5+10;
int a[nmax];

#define ll long long

int main(){
    freopen("input.txt","r",stdin);
    int T;
    cin>>T;
    while(T--){
        int n,k;
        cin>>n>>k;
        for(int i=0;i<n;i++)cin>>a[i];
        ll ans=-1e12;

        ll Fst=max(0,n-2*k);
        for(int i=Fst;i<n;i++)for(int j=i+1;j<n;j++){
            ans=max(ans,1ll*(i+1)*(j+1)-1ll*k*(a[i]|a[j]));
        }
        cout<<ans<<endl;
    }
}
\end{lstlisting}

\section{
    C.Mikasa
}

\subsection{题意}
\paragraph{
    原题\href{http://codeforces.com/contest/1554/problem/C}{链接}\\
    给出两个数\(m,n\),求数列\(n\oplus 1,n\oplus 2,...,n\oplus m\)的MEX值。\\
    规定数列的MEX为该数列与自然数列补集的最小值,例如MEX(0,1,2,4) = 3,MEX(1,2021) = 0。
}

\subsection{思路}
\begin{paragraph}
    若\(a\oplus b=c\),则\(a\oplus c=b\)。\\
    所以只要从0开始递增遍历所有的c,当第一个大于m的c出现时,就是答案,时间复杂度\(\mathcal{O}(n)\)。\\
    但是题目中n和m实在是太大(高达\(10^9\)),所以我们还需要继续优化。\\
    由于异或运算能够改变指定二进制位的值,所以我们位运算的角度考虑如何求c的最小值:\\
    ·若n的二进制位数多于m(即n>m),则答案为0。\\
    ·若n的二进制位数少于或等于m,要使\(n\oplus c\)大于m,则需要将n的二进制0位从高到低逐个变成1位,直到其大于m为止。此时参与异或运算的c即为最小值。c的值为变零位是1,其他位补零。\\
    例:\(n=101010_{(2)},m=110101_{(2)},n\oplus c=111010_{(2)},c=10000_{(2)}=16_{(10)}\)\\
    此时,时间复杂度为\(\mathcal{O}(log(n))\)。\\
    水平叭太行感觉写得好绕QAQ,可以参考原题解,\href{http://codeforces.com/blog/entry/93321}{链接}。
\end{paragraph}

\subsection{代码}
\begin{lstlisting}
    #include"bits/stdc++.h"

using namespace std;

int main(){
    freopen("input.txt","r",stdin);
    int t;
    cin>>t;
    while(t--){
        int n,m;
        cin>>n>>m;
        m++;
        int ans=0;
        for(int i=30;i>=0&&n<m;i--)if((n>>i&1)!=(m>>i&1)){
                ans|=1<<i;
                n|=1<<i;
            }
        cout<<ans<<endl;
    }
}
\end{lstlisting}

\section{
    D.Diane
}

\subsection{题意}
\begin{paragraph}
    原题\href{http://codeforces.com/contest/1554/problem/D}{链接}\\
    构造一个长度为n的字符串,使得它的所有非空子字符串的出现次数为奇数次。
\end{paragraph}

\subsection{思路}

\begin{paragraph}
    首先要正确理解题意,题目中是子字符串的出现次数,不是子字符串中字母出现次数(我一开始没看懂题,加上具有诱导的样例,就再一次泪目了QAQ),此题题解极为精妙,看完后我直呼“妙啊”。\\
    首先考虑一个字母的情况。在字符串\(\underbrace{aaa...a}_{n} \)中,"a"出现了n次,"aa"出现了n-1次,"aaa"出现了n-2次,以此类推,若n为奇数,则出现次数依次是奇数,偶数,奇数……\\
    由于奇数+偶数=奇数,所以在构造数列时,设\(k=\lfloor \frac{n}{2} \rfloor \):\\
    ·若n为偶数,则构造数列为\(\underbrace{a...a}_{k-1} b\underbrace{a...a}_{k}\),前半部分"a","aa","aaa"...出现次数为偶,奇,偶,奇...,后半部分对应字串出现次数为奇,偶,奇,偶...,相加后的总次数均为奇数次。\\
    ·若n为奇数,则构造数列为\(\underbrace{a...a}_{k-1} bc\underbrace{a...a}_{k-2}\),原理同上。\\
    按上述方法构造即为答案。\\
    妙啊~~~\\
    在验证答案时,由于子串数量巨大,oj用了一个后缀自动机感兴趣的同学可以去原题解看看,我并没有看懂QAQ。
\end{paragraph}

\subsection{代码}
\begin{lstlisting}
    #include"bits/stdc++.h"

using namespace std;

int main(){
    freopen("input.txt","r",stdin);
    int T;
    cin>>T;
    while(T--){
        int n;
        cin>>n;
        int k;
        if(n==1)cout<<'a'<<endl;
        else if(n%2==0){
            k=(n-1)/2;
            for(int i=0;i<k;i++)cout<<'a';
            cout<<'b';
            for(int i=0;i<k+1;i++)cout<<'a';
            cout<<endl;
        }
        else{
            k=(n-2)/2;
            for(int i=0;i<k;i++)cout<<'a';
            cout<<"bc";
            for(int i=0;i<k+1;i++)cout<<'a';
            cout<<endl;
        }
    }
}
\end{lstlisting}

\section{
    E.You
}

\subsection{题意}
\begin{paragraph}
    原题\href{http://codeforces.com/contest/1554/problem/E}{链接}\\
    给出一棵树,按不同的顺序依次删除其每一个节点,记录删除时该节点连接的其他节点的数量,从节点1到n组成数列,求该数列的最大公约数并对998244353取模,统计不同删除顺序下产生的数列的最大公约数相同的个数。
\end{paragraph}

\subsection{思路}
\begin{paragraph}
    很难,我还没有完全看懂,决定先去玩一会儿稻妻,待会儿再写-v-
\end{paragraph}
\end{document}