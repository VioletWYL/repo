\documentclass[cyan,pad,cn]{elegantnote}

\usepackage{amsmath}
\usepackage{listings}
\usepackage{xcolor}

\definecolor{CPPLight}  {HTML} {686868}
\definecolor{CPPSteel}  {HTML} {888888}
\definecolor{CPPDark}   {HTML} {262626}
\definecolor{CPPBlue}   {HTML} {4172A3}
\definecolor{CPPGreen}  {HTML} {487818}
\definecolor{CPPBrown}  {HTML} {A07040}
\definecolor{CPPRed}    {HTML} {AD4D3A}
\definecolor{CPPViolet} {HTML} {7040A0}
\definecolor{CPPGray}  {HTML} {B8B8B8}
\lstset{
    columns=fixed,       
    numbers=left,                                        % 在左侧显示行号
    frame=none,                                          % 不显示背景边框
    backgroundcolor=\color[RGB]{245,245,244},            % 设定背景颜色
    keywordstyle=\color[RGB]{40,40,255},                 % 设定关键字颜色
    numberstyle=\footnotesize\color{darkgray},           % 设定行号格式
    commentstyle=\it\color[RGB]{0,96,96},                % 设置代码注释的格式
    stringstyle=\rmfamily\slshape\color[RGB]{128,0,0},   % 设置字符串格式
    showstringspaces=false,                              % 不显示字符串中的空格
    language=c++,                                        % 设置语言
    morekeywords={alignas,continute,friend,register,true,alignof,decltype,goto,
    reinterpret_cast,try,asm,defult,if,return,typedef,auto,delete,inline,short,
    typeid,bool,do,int,signed,typename,break,double,long,sizeof,union,case,
    dynamic_cast,mutable,static,unsigned,catch,else,namespace,static_assert,using,
    char,enum,new,static_cast,virtual,char16_t,char32_t,explict,noexcept,struct,
    void,export,nullptr,switch,volatile,class,extern,operator,template,wchar_t,
    const,false,private,this,while,constexpr,float,protected,thread_local,
    const_cast,for,public,throw,std},
    emph={map,set,multimap,multiset,unordered_map,unordered_set,
    unordered_multiset,unordered_multimap,vector,string,list,deque,
    array,stack,forwared_list,iostream,memory,shared_ptr,unique_ptr,
    random,bitset,ostream,istream,cout,cin,endl,move,default_random_engine,
    uniform_int_distribution,iterator,algorithm,functional,bing,numeric,},
    emphstyle=\color{CPPViolet}, 
}

\title{
    补题笔记·ACWing周赛\#11
}
\author{唐冬   \LaTeX}

\begin{document}

\maketitle

\section{
    A.计算abc
}

\subsection{题意}
\paragraph{原题\href{https://www.acwing.com/problem/content/3798/}{链接}}

\subsection{思路}
\paragraph{
    在比赛的时候,由于我自己眼瞎,没有看到题目中说全部数据为正整数,于是写了一个整数范围的解QAQ,正确的解法是先从小到大排序,最大的数一定是三个数的和,再分别做减法,就能求出a,b,c三个数的值了。
    \\我的解法是,先将给出的四个数求和
}
\begin{equation}
    \begin{split}
        Sum_{2} &={} a+b+c+Sum_{1} \\
        &={}3a+3b+3c 
    \end{split} 
\end{equation}


\paragraph{
    再除以三,得到三个数之和,在进行匹配,找出题目给出的四个数中不是和的三个,再排序后分别做差,就能求出答案了。
    \\下次一定认真读题-v-
}

\subsection{代码}

\begin{lstlisting}
    #include"bits/stdc++.h"
  
    using namespace std;
    
    int main(){
        freopen("input.txt","r",stdin);
        string s;
        cin>>s;
        int ans=-1;
        for(int i=1;i<=s.size();i++){
            string s2;
            for(int j=0;j<i;j++){
                int p=j;
                while(p<s.size()){s2+=s[p];p+=i;}
                int x=1,y=0,z=1;
                p=1;
                while(p<s2.size()){
                    if(s2[p]==s2[p-1])z++;
                    else {
                        y=max(y,z);
                        if(x<y)swap(x,y);
                        z=1;
                    }
                    p++;
                }
                y=max(y,z);
                if(x<y)swap(x,y);
                z=0;
                ans=max(ans,x*y);
            }
        }
        cout<<ans<<endl;
    } 
\end{lstlisting}

\section{
    B.凑平方
}

\subsection{题意}

\paragraph{原题\href{https://www.acwing.com/problem/content/3799/}{链接}} 

\subsection{思路}

\paragraph{
    题目中说每次操作可删除一位数字,所以主体思路是递归遍历模拟删除数字的每一位,类似于遍历连续子串,再判断是否为平方数,记录递归深度(删除的数的个数),最小的记录就是答案。
    \\那么如何判断平方数呢?打表数组开不下,可以先开平方,再平方,判断是否和原数相等(利用C++截断特性);
    \\如果遍历全部情况,会导致超时。需要三个剪枝:
    \\·找到平方数后直接返回,因为再次递归的答案一定不是最优解。
    \\·当前深度不小于记录值时直接返回。
    \\·当前数遍历完成后,当再次遇到时直接返回,因为我们已经遍历过它和它的全部子集了。
    \\需要额外注意,题目中说需要删除无用的前导零(我因为没有在意这个,喜-2,血亏QAQ)
}

\subsection{代码}
\begin{lstlisting}
    #include"bits/stdc++.h"

    using namespace std;
    
    #define ll long long
    
    map<int,bool>Hc;
    
    const int nmax=1e9+1;
    bool Hash(ll x){
        ll k=(ll)sqrt(x);
        if(x!=0&&k*k==x)return 1;
        return 0;
    }
    
    int ans;
    
    void dfs(ll a,int u){
        if(a==0||Hc.count(a)>0)return;
        Hc[a]=1;
        u++;
        if(u>=ans)return;
        int n=0;
        for(ll i=1;i<=1e9;i*=10){
            if(a>(i-1))n++;
            else break;
        }
        if(n==1){if(Hash(a))ans=u;return;}
        
        bool ok=0;
        for(int i=0;i<n;i++){
            ll k=a%(ll)pow(10,i);
            k+=(a/(ll)pow(10,i+1))*(ll)pow(10,i);
            if(Hash(k)){ok=1;
            int n2=0;
            for(ll j=1;j<=1e9;j*=10){
            if(k>(j-1))n2++;
            else break;
        }
            ans=min(ans,u+n-n2-1);
            return;}
            else dfs(k,u);
        }
    }
    
    int main(){
        //freopen("input.txt","r",stdin);
        int T;
        cin>>T;
        while(T--){
            Hc.clear();
            ans=100;
            ll a;
            cin>>a;
            
            if(Hash(a))ans=0;
            else dfs(a,0);
    
            bool ok=0;
            if(ans<100)cout<<ans<<endl;
            else cout<<"-1\n";
        }
    }
\end{lstlisting}

\section{
    C.最大化最短路
}

\subsection{题意}
\paragraph{原题\href{https://www.acwing.com/problem/content/3800/}{链接}}
\subsection{思路}
\paragraph{
    题中说固定起点终点,要求选点连接并求出最短路的最大值,那么最短路可能有三种情况:
    \\情况1:从1点出发到n点并经过新路;
    \\情况2:从n点出发到1点并经过新路;
    \\情况3:不经过新路的最短路(1,n出发都一样)。
    \\考虑情况1和情况2,用BFS分别求出从1,
    n出发到连接点的最短路,记做
    \(d_{0,p},d_{1,p}\),总路长即为:
    }
    \begin{equation}
    S_{总}=dist_{0,x}+dist_{1,y}+1
    \end{equation}
\paragraph{
        再通过遍历x,y求出最大值。
        在求最大值的过程中,遍历到x,y时会遇到两种情况,x->y和y->x;为减少遍历次数,可以先进行排序:
}

\begin{equation}
    d_{0,x}+d_{1,y}<d_{1,x}+d_{0,y}
\end{equation}
{移项,得:}
\begin{equation}
d_{0,x}-d_{1,x}<d_{0,y}-d{1,y}
\end{equation}

\paragraph{
    排序后可顺序遍历一次在O(n)内求出最大值。
}

\subsection{代码}
\begin{lstlisting}
    #include"bits/stdc++.h"

    using namespace std;
    
    const int nmax=2e6+10;
    
    int n,m,k;
    int Node[nmax],          //连接点
        dist[2][nmax];      //路径记录
    
    vector<int>Link[nmax];   //邻接表
    
    void bfs(int fst){
        queue<int>q;
            q.push(fst);
            int kk=fst==1?0:1;
            dist[kk][fst]=0;
            while(!q.empty()){
                int p=q.front();
                int v=dist[kk][p];
                q.pop();
                for(int i=0;i<Link[p].size();i++)if(dist[kk][Link[p][i]]<0){
                    dist[kk][Link[p][i]]=v+1;
                    //printf("dist[%d][%d]=%d\n",kk,Link[p][i],v+1);
                    q.push(Link[p][i]);
                }
            }
    }
    
    int main(){
        //freopen("input.txt","r",stdin);
        cin>>n>>m>>k;
        for(int i=0;i<k;i++){
            int a;
            scanf("%d",&a);
            Node[i]=a;
        }
        for(int i=0;i<m;i++){
            int a,b;
            scanf("%d%d",&a,&b);
            Link[a].push_back(b);
            Link[b].push_back(a);
        }
        memset(dist,-1,sizeof(dist));
        bfs(1);
        bfs(n);
    
        sort(Node,Node+k,[&](int a,int b){//c++17新特性
            return dist[0][a]-dist[1][a]<dist[0][b]-dist[1][b];
        });
    
        int res=0,fst=dist[0][Node[0]];
        for(int i=1;i<k;i++){
            int v=dist[1][Node[i]];
            res=max(res,v+fst+1);
            fst=max(fst,dist[0][Node[i]]);
        }
    
        cout<<min(dist[0][n],res)<<endl;
    }
\end{lstlisting}

{在BFS求最短路时我更习惯用迭代器,大雪菜老师用的是数组,可以学习一下大佬的代码(*¯︶¯*)。\href{https://www.acwing.com/activity/content/code/content/1592514/}{链接}}

\begin{lstlisting}
    #include <iostream>
#include <cstring>
#include <algorithm>

using namespace std;

const int N = 200010, M = 400010;

int n, m, k;
int h[N], e[M], ne[M], idx;
int dist1[N], dist2[N];
int q[N], p[N];

void add(int a, int b)  // 添加一条边a->b
{
    e[idx] = b, ne[idx] = h[a], h[a] = idx ++ ;
}

void bfs(int start, int dist[])
{
    memset(dist, 0x3f, N * 4);
    dist[start] = 0;
    int hh = 0, tt = 0;
    q[0] = start;

    while (hh <= tt)
    {
        int t = q[hh ++ ];

        for (int i = h[t]; ~i; i = ne[i])
        {
            int j = e[i];
            if (dist[j] > dist[t] + 1)
            {
                dist[j] = dist[t] + 1;
                q[ ++ tt] = j;
            }
        }
    }
}


int main()
{
    scanf("%d%d%d", &n, &m, &k);
    memset(h, -1, sizeof h);
    for (int i = 0; i < k; i ++ ) scanf("%d", &p[i]);
    while (m -- )
    {
        int a, b;
        scanf("%d%d", &a, &b);
        add(a, b), add(b, a);
    }

    bfs(1, dist1);
    bfs(n, dist2);

    sort(p, p + k, [&](int a, int b) {
        return dist1[a] - dist2[a] < dist1[b] - dist2[b];
    });

    int res = 0, x = dist1[p[0]];
    for (int i = 1; i < k; i ++ )
    {
        int t = p[i];
        res = max(res, dist2[t] + x + 1);
        x = max(x, dist1[t]);
    }

    printf("%d\n", min(res, dist1[n]));
    return 0;
}

\end{lstlisting}

\end{document}